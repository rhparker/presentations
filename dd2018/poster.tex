%%%%%%%%%%%%%%%%%%%%%%%%%%%%%%%%%%%%%%
% Multiplicative domain poster
% Created by Nathaniel Johnston
% August 2009
% http://www.nathanieljohnston.com/2009/08/latex-poster-template/
%%%%%%%%%%%%%%%%%%%%%%%%%%%%%%%%%%%%%%

\documentclass[final]{beamer}
\usepackage[scale=1.24]{beamerposter}
\usepackage{graphicx}			% allows us to import images

%-----------------------------------------------------------
% Custom commands that I use frequently
%-----------------------------------------------------------

\newcommand{\bb}[1]{\mathbb{#1}}
\newcommand{\cl}[1]{\mathcal{#1}}
\newcommand{\fA}{\mathfrak{A}}
\newcommand{\fB}{\mathfrak{B}}
\newcommand{\Tr}{{\rm Tr}}
\newtheorem{thm}{Theorem}

%-----------------------------------------------------------
% Define the column width and poster size
% To set effective sepwid, onecolwid and twocolwid values, first choose how many columns you want and how much separation you want between columns
% The separation I chose is 0.024 and I want 4 columns
% Then set onecolwid to be (1-(4+1)*0.024)/4 = 0.22
% Set twocolwid to be 2*onecolwid + sepwid = 0.464
%-----------------------------------------------------------

\newlength{\sepwid}
\newlength{\onecolwid}
\newlength{\twocolwid}
\setlength{\paperwidth}{48in}
\setlength{\paperheight}{36in}
\setlength{\sepwid}{0.024\paperwidth}
\setlength{\onecolwid}{0.22\paperwidth}
\setlength{\twocolwid}{0.464\paperwidth}
\setlength{\topmargin}{-0.5in}
\usetheme{confposter}
\usepackage{exscale}

%-----------------------------------------------------------
% The next part fixes a problem with figure numbering. Thanks Nishan!
% When including a figure in your poster, be sure that the commands are typed in the following order:
% \begin{figure}
% \includegraphics[...]{...}
% \caption{...}
% \end{figure}
% That is, put the \caption after the \includegraphics
%-----------------------------------------------------------

\usecaptiontemplate{
\small
\structure{\insertcaptionname~\insertcaptionnumber:}
\insertcaption}

%-----------------------------------------------------------
% Define colours (see beamerthemeconfposter.sty to change these colour definitions)
%-----------------------------------------------------------

\setbeamercolor{block title}{fg=ngreen,bg=white}
\setbeamercolor{block body}{fg=black,bg=white}
\setbeamercolor{block alerted title}{fg=white,bg=dblue!70}
\setbeamercolor{block alerted body}{fg=black,bg=dblue!10}

%-----------------------------------------------------------
% Name and authors of poster/paper/research
%-----------------------------------------------------------

\title{Stability of Double Pulse Solutions to the 5th order Korteweg de Vries (KdV) Equation, a Numerical Approach}
\author{Ross Parker, Bj\"{o}rn Sandstede}
\institute{Division of Applied Mathematics, Brown University}

%-----------------------------------------------------------
% Start the poster itself
%-----------------------------------------------------------
% The \rmfamily command is used frequently throughout the poster to force a serif font to be used for the body text
% Serif font is better for small text, sans-serif font is better for headers (for readability reasons)
%-----------------------------------------------------------

\begin{document}
\begin{frame}[t]
  \begin{columns}[t]												% the [t] option aligns the column's content at the top
    \begin{column}{\sepwid}\end{column}			% empty spacer column

% first column
    \begin{column}{\onecolwid}
      \begin{alertblock}{Goal}
        \rmfamily{
        Examine the spectral stability of the linearization of the 5th order KdV equation about
        double pulse solutions.
        }
      \end{alertblock}

      \vskip2ex

      \begin{block}{Fifth-order KdV Equation}
        \rmfamily{
            \[ u_t = u_{xxxxx} - u_{xxx} + u u_x \]
          Applications
          \begin{itemize}
            \item Capillary gravity water waves
            \item Plasma waves
            \item Laser optics
          \end{itemize}
        }
      \end{block}

      \begin{block}{Traveling Wave Solutions}
        \rmfamily{
          Using traveling wave ansatz $u(x, t) = u(x - ct, t)$
            \[ u_t = u_{xxxxx} - u_{xxx} - c u_x + u u_x \]
          
          Equilibrium solution satisfies 5th order ODE 
            \[ u_{xxxxx} - u_{xxx} - c u_x + u u_x = 0 \]

          Since we seek solutions which decay to 0 at $\pm \infty$, 
          integrate once to get 4th order ODE 
            \[u_{xxxx} - u_{xx} - cu + u^2 = 0\]

          Equation is Hamiltonian with conserved quantity
          \[
            H = u_x u_{xxx} - \frac{1}{2}u_x^2 - \frac{1}{2}u_{xx}^2 + \frac{c}{2}u^2 - \frac{1}{3}u^3
          \]
        }
      \end{block}

      \vskip2ex

      \begin{block}{Homoclinic Orbits}
        \rmfamily{
          Linearization about zero solution
            \[v_{xxxx} - v_{xx} - c v = 0 \]
          Eigenvalues of linearization
            \[ \lambda = \pm \sqrt{ \frac{1 \pm \sqrt{1 - 4c } }{ 2} } \]
          2D stable manifold, 2D unstable manifold\\
          \vskip2ex
          \textbf{Theorem.} \emph{For $c>0$, a homoclinic orbit (single pulse solution) $u(x)$ exists which is smooth, even, and decays exponentially to 0 as $x \rightarrow \pm \infty.$}
        }
      \end{block}
 
    \end{column}

\begin{column}{\sepwid}\end{column}     % empty spacer column

% second column
    \begin{column}{\onecolwid}

      \begin{block}{Bifurcation}
      At $c = 1/4$, eigenvalues of linearization about zero solution undergo a bifurcation.
        \begin{figure}
          \begin{center}
            \includegraphics[width=0.8\textwidth]{images/eigbifurcation}
            \caption{}
            \label{fig:bifurcation}
          \end{center}
        \end{figure}
        For $c > 1/4$, $\lambda = \pm \alpha \pm \beta i$. Single pulse solution has exponential decay rate $\alpha$, tail oscillations with frequency $\beta$.
      \end{block} 
        \begin{figure}
          \begin{center}
            \includegraphics[width=0.8\textwidth]{images/singlepulsemagright}
            \caption{}
            \label{fig:singlepulsemagright}
          \end{center}
        \end{figure}

    \end{column}

\begin{column}{\sepwid}\end{column}     % empty spacer column

% third column
    \begin{column}{\onecolwid}

      \begin{block}{Stuff}
        \rmfamily{
          \begin{center}
            \[ u_t = u_{xxxxx} - u_{xxx} + u u_x \]
          \end{center}
          Applications
          \begin{itemize}
            \item Capillary gravity water waves
            \item Plasma waves
            \item Laser optics
          \end{itemize}
        }
      \end{block} 
  
    \end{column}

  \begin{column}{\sepwid}\end{column}			% empty spacer column

  \begin{column}{\onecolwid}
    \begin{block}{Conclusions and Outlook}
      \rmfamily{This characterization provides a simple way to find all unitarily-correctable codes for unital channels and even some codes for non-unital channels. General correctable subsystems can be characterized in terms of algebras that are analogous to the multiplicative domain, though in general it is not clear how to calculate them -- further research in this area would be of great interest.}
    \end{block}
    \vskip2ex
    \begin{block}{For Further Information}
      \small{\rmfamily{For the details of our work:
      \begin{itemize}
        \item Choi, M.-D., Johnston, N., and Kribs, D. W.. Journal of Physics A: Mathematical and Theoretical \textbf{42}, 245303 (2009).
        \item Johnston, N., and Kribs, D. W., \emph{Generalized Multiplicative Domains and Quantum Error Correction} (2009, preprint).
      \end{itemize}
      \vspace{0.1in}\noindent Preprints and this poster can be downloaded from:
      \begin{itemize}
        \item www.arxiv.org
        \item www.nathanieljohnston.com
      \end{itemize}}}
    \end{block}
    \vskip2ex
    \begin{block}{References}
      \small{\rmfamily{\begin{thebibliography}{99}
      \bibitem{KLPL06} D.~W. Kribs, R. Laflamme, D. Poulin, M. Lesosky, Quantum Inf. \& Comp. \textbf{6} (2006), 383-399.
      \bibitem{zanardi97} P. Zanardi, M. Rasetti, Phys. Rev. Lett. \textbf{79},  3306 (1997).
      \bibitem{KS06} D.~W. Kribs, R.~W. Spekkens, Phys. Rev. A \textbf{74}, 042329 (2006).
      \bibitem{Cho74} M.-D. Choi, Illinois J. Math., \textbf{18} (1974), 565-574.
      \bibitem{Dav96} K.~R. Davidson, \emph{$C^*$-algebras by example}, Fields Institute Monographs, 6. American Mathematical Society, Providence, RI, 1996.
      \end{thebibliography}}}
    \end{block}
    \vskip2ex
    \begin{block}{Acknowledgements}
      \small{\rmfamily{M.-D.C. was supported by an NSERC Discovery Grant. N.J. was supported by an NSERC Canada Graduate Scholarship and the University of Guelph Brock Scholarship. D.W.K. was supported by an NSERC Discovery Grant and Discovery Accelerator Supplement, an Ontario Early Researcher Award, and CIF, OIT.}} \\
      \vspace{0.5in}
      \begin{center}
        \begin{tabular}{ccc}
          \includegraphics[width=3in]{TO.png} & \hspace{1.5in} & \includegraphics[width=3in]{guelph.png}
        \end{tabular}
      \end{center}
    \end{block}
  \end{column}
  \begin{column}{\sepwid}\end{column}			% empty spacer column
 \end{columns}
\end{frame}
\end{document}

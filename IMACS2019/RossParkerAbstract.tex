% waves2015abstract_template.tex
% by Gino Biondini and Thiab Taha, 2014/8/18
%
% Use this file to submit abstracts for talks.

% Please do not modify the TeX commands below!
% The abstracts will be reformatted and combined in the conference's book of abstracts.

\documentclass[12pt]{article}
\textwidth 6.5in
\textheight 8.5in
\renewcommand\floatpagefraction{0.9}
\renewcommand{\baselinestretch}{1.0}
\topmargin 0.3in
\parskip 10pt
\parindent 0pt
\evensidemargin 0mm
\oddsidemargin 0mm
\renewcommand{\refname}{\normalsize\bf References} 
\makeatletter
\renewcommand\section{\@startsection {section}{1}{\z@}%
  {-2.4ex \@plus -1ex \@minus -.2ex}{1.0ex \@plus.1ex}%
  {\normalfont\large\bf}}
\def\title#1{{\large\bf #1}} 
\def\author#1#2#3{\vglue1ex{\bf #1}\\[0.4ex]{\small #2}\\[0.2ex]\textit{\small #3}}
\long\def\abstract#1{\par\bigskip\centerline{\bf Abstract}\par\smallskip}
\makeatother

\begin{document}
\pagestyle{empty}
\begin{center}

% Please fill in the session information as appropriate in the fields below.

\title{Spectral stability of multi-pulses via the Krein matrix}

% Use this format if two or more authors share the same address
\author{Ross Parker$^*$}
  {Division of Applied Mathematics\\Brown University}
  {ross\_parker@brown.edu} 

% Use the asterisk for presenters 
\author{Todd Kapitula}
  {Deparment of Mathematics and Statistics\\Calvin College}
  {tmk5@calvin.edu}

% Use this command to insert additional authors if necessary
\author{Bj{\"o}rn Sandstede}
  {Division of Applied Mathematics\\Brown University}
  {bjorn\_sandstede@brown.edu}

\end{center}

\abstract

The Chen-Mckenna suspension bridge equation is a nonlinear PDE which is 2nd order in time and is used to model traveling waves on a suspended beam. For certain parameter regimes, it admits multi-pulse traveling wave solutions, which are small perturbations of the stable, primary pulse solution. Linear stability of these multi-pulse solutions is determined by eigenvalues near the origin representing the interaction between the individual pulses. Linearization about these multi-pulse solutions yields a quadratic eigenvalue problem. To study this problem, we use a reformulated version of the Krein matrix, which was presented by Todd Kapitula in a previous talk. Using an appropriate leading order expansion of the Krein matrix, we are able to give analytical criteria for the stability of these multi-pulse solutions. We also present numerical results to support our analysis. 

\end{document}

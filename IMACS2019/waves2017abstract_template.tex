% waves2015abstract_template.tex
% by Gino Biondini and Thiab Taha, 2014/8/18
%
% Use this file to submit abstracts for talks.

% Please do not modify the TeX commands below!
% The abstracts will be reformatted and combined in the conference's book of abstracts.

\documentclass[12pt]{article}
\textwidth 6.5in
\textheight 8.5in
\renewcommand\floatpagefraction{0.9}
\renewcommand{\baselinestretch}{1.0}
\topmargin 0.3in
\parskip 10pt
\parindent 0pt
\evensidemargin 0mm
\oddsidemargin 0mm
\renewcommand{\refname}{\normalsize\bf References} 
\makeatletter
\renewcommand\section{\@startsection {section}{1}{\z@}%
  {-2.4ex \@plus -1ex \@minus -.2ex}{1.0ex \@plus.1ex}%
  {\normalfont\large\bf}}
\def\title#1{{\large\bf #1}} 
\def\author#1#2#3{\vglue1ex{\bf #1}\\[0.4ex]{\small #2}\\[0.2ex]\textit{\small #3}}
\long\def\abstract#1{\par\bigskip\centerline{\bf Abstract}\par\smallskip}
\makeatother

\begin{document}
\pagestyle{empty}
\begin{center}

% Please fill in the session information as appropriate in the fields below.

\title{Title of the presentation}

% Use this format if two or more authors share the same address
\author{First author name and second author name}
  {Address of the first and second author\\Second line of the address, if necessary}
  {first.author@university.edu and second.author@university.edu} 

% Use the asterisk for presenters 
\author{Third author's name$^*$}
  {Address (add lines as above if necessary)}
  {third.author@university.edu}

% Use this command to insert additional authors if necessary
\author{Fourth author name}
  {Address (add lines as above if necessary)}
  {fourth.author@university.edu}

\end{center}

\abstract

Please remember that abstracts must be no more than one page in length.
Also, please refrain from modifying the \LaTeX\ formatting, as abstracts will be reformatted and combined in the conference's book of abstracts.
Below is a sample abstract.

The class of complex modified Korteweg-de Vries (CMKdV) equations has many
applications. One form of the CMKdV equation has been used to create models 
for the nonlinear evolution of plasma waves, for the propagation of transverse 
waves in a molecular chain, and for a generalized elastic solid.
Another form of the CMKdV equation has been used for the traveling-wave and
for a double homoclinic orbit \cite{Herbst_Ablowitz_Ryan}.

In this paper we introduce sequential and parallel split-step Fourier methods
for numerical simulations of the above equation.
The parallel methods are implemented on the Origin 2000 multiprocessor computer.
Our numerical experiments have shown that the finite difference and the
inverse scattering methods give accurate results and considerable speedup.

\begin{thebibliography}{1}

\bibitem{Herbst_Ablowitz_Ryan} B. M. Herbst, M. J. Ablowitz and E. Ryan,
Numerical homoclinic instabilities and the complex modified Korteweg-de Vries equation,
{\sl Comput. Phys. Commun.}, {\bf 65} (1991), 137-142.

\end{thebibliography}

\end{document}
